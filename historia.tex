% temas de heuristica
\documentclass{article}
\usepackage[spanish]{babel} % Carga el paquete para el idioma español
\usepackage[T1]{fontenc}    % Codificación de la fuente
\usepackage[utf8]{inputenc} % Codificación de entrada para caracteres UTF-8
\usepackage{graphicx} %LaTeX package to import graphics
\usepackage{natbib}

\title{Heuristica y Matemáticas}
\author{Brigida Carvajal}
\date{Septiembre - 2024}
\begin{document}



\maketitle

\tableofcontents
\newpage
\section{Introducción}
La heurística es una estrategia que se utiliza para encontrar soluciones aproximadas a problemas complejos mediante métodos no rigurosos pero efectivos. En matemáticas, la heurística ha sido fundamental para el descubrimiento y la resolución de problemas a lo largo de la historia.
\begin{figure}[h!]
    \centering
    \includegraphics[width=0.5\textwidth]{th.jpg}
    \caption{Heuristica}
    \label{fig:mi_imagen}
\end{figure}
\section{Ejemplos Históricos}
\subsection{Arquímedes y el Principio de Flotación}
Arquímedes utilizó el principio de desplazamiento de agua para resolver el problema. Al sumergir la corona y medir el volumen de agua desplazada, pudo comparar la densidad de la corona con la del oro puro.
\subsection{Newton y el Cálculo}
Newton desarrolló el cálculo diferencial e integral como herramientas heurísticas para modelar el cambio y el movimiento.
\subsection{Gauss y la Suma de una Serie Aritmética}
Gauss, a una edad temprana, descubrió que sumando los números en pares (1+100, 2+99, etc.), podía simplificar el cálculo.\citep{krick1991algorithm}
\subsection{Pólya y la Resolución de Problemas}
George Pólya desarrolló un enfoque sistemático para la resolución de problemas, que incluye comprender el problema, diseñar un plan, ejecutar el plan y revisar la solución.
\section{Conclusión}
La heurística ha sido una herramienta poderosa en la historia de las matemáticas, permitiendo a los matemáticos descubrir soluciones innovadoras y efectivas a problemas complejos. Estos ejemplos históricos demuestran cómo la heurística puede guiar el pensamiento creativo y la resolución de problemas en matemáticas.
\citep{krick2001sharp}
\bibliographystyle{alpha}
\bibliography{referencias}
\end{document}